\subsection{Evolutionary Algorithm Module}

The evolutionary algorithm module introduces a novel approach to agent specialization evolution through \textit{prompt-level genetic operations}. Unlike traditional evolutionary algorithms that operate on bit strings or numerical parameters, our implementation performs crossover and mutation directly in the semantic space of natural language specializations, leveraging Large Language Models (LLMs) as genetic operators.

\subsubsection{Semantic Evolution Framework}

The core innovation lies in treating agent specialization prompts as evolvable genotypes. Each agent possesses a \texttt{specific\_prompt} field that defines its behavioral specialization (e.g., "Focus on mathematical optimization tools" or "Create data visualization utilities"). This specialization prompt, not the agent's reflection history, serves as the genetic material subject to evolution. The \texttt{reflection\_history} serves as phenotypic memory, recording the agent's past observations and decisions, but remains unchanged during evolution. Evolution occurs through LLM-mediated genetic operations on the specialization prompts that preserve semantic coherence while introducing meaningful variation.

\textbf{Fitness Evaluation:} Agent fitness is determined by the average Tool Complexity Index (TCI) of all tools the agent has successfully created. The TCI metric combines code complexity, interface sophistication, and compositional reuse, providing an objective measure of an agent's contribution to ecosystem capability. Agents with no successfully created tools receive a fitness score of 0.0.

\textbf{Selection Mechanism:} The algorithm employs truncation selection, eliminating the bottom 20\% of agents based on fitness ranking while maintaining a minimum population size threshold. This approach provides consistent selection pressure while preventing population collapse in small experimental groups.

\subsubsection{Prompt-Level Genetic Operations}

\textbf{Crossover (Sexual Reproduction):} When two parent agents are selected for reproduction, their specialization prompts are combined through an LLM-mediated process. The system prompt instructs the LLM to "create a new agent specialization that combines the best aspects of both parents," while the user prompt provides the parent specializations and their performance metrics. This generates semantically coherent hybrid specializations that blend parental traits. Temperature is set to 0.7 to balance creativity with consistency.

\textbf{Mutation (Asexual Reproduction):} Single-parent reproduction creates variations through semantic mutation. The LLM is prompted to "create a related but distinct specialization that explores a different angle or approach" based on the parent's specialization. Higher temperature (0.8) encourages greater exploration while maintaining thematic coherence with the parent specialization.

\subsubsection{Integration with Boids Coordination}

The evolutionary algorithm operates as a complementary mechanism to boids rules rather than a replacement. During normal operation, agents follow local neighborhood coordination rules (separation, alignment, cohesion) for immediate tool creation decisions. Evolution intervenes periodically (every $n$ rounds, configurable) to update agent specializations based on accumulated performance data.

This dual-mechanism approach enables both short-term local coordination and long-term global optimization. Boids rules ensure immediate ecosystem coherence and diversity, while evolutionary pressure drives specialization refinement and capability enhancement over multiple generations.

\subsubsection{Evolutionary Dynamics}

\textbf{Timing:} Evolution triggers at configurable intervals (default: every 5 rounds) during the final rounds of experiments to allow sufficient performance data accumulation. The condition \texttt{round\_num \% evolution\_frequency == 0} determines trigger timing.

\textbf{Population Management:} New agents are created with evolved specializations but retain all other characteristics (Azure client, tool registry access, environment capabilities). Agent IDs follow the pattern \texttt{EvoAgent\_Gen\{N\}\_\{i\}\_\{Type\}} where N is the generation number, i is the agent index, and Type indicates "Cross" (crossover) or "Mut" (mutation).

\textbf{Fallback Mechanisms:} Robust error handling ensures system stability when LLM calls fail. Failed crossover attempts default to descriptive hybrid specializations, while failed mutations default to variant descriptions. This prevents evolution failure from disrupting the broader experimental framework.

\subsubsection{Data Collection and Analysis}

The system maintains comprehensive evolutionary records including generation numbers, population sizes, fitness distributions, eliminated agent IDs, and complete specialization genealogies. Evolution history enables post-hoc analysis of specialization trajectories, fitness landscapes, and convergence patterns.

Performance metrics track complexity improvement over generations, total agents eliminated and created, and diversity measures across the evolving population. This data supports both immediate experimental analysis and broader theoretical insights into semantic evolution dynamics.

\subsubsection{Experimental Validation}

The modular design enables systematic comparison across four experimental conditions: (1) boids coordination only, (2) evolutionary selection only, (3) combined boids and evolution, and (4) control condition with neither mechanism. This factorial design isolates the contributions of local coordination and global selection pressure to collective intelligence emergence.

By introducing evolutionary pressure into the semantic space of agent specializations while preserving local coordination mechanisms, this approach represents a novel synthesis of swarm intelligence and evolutionary computation principles, enabling investigation of how global selection pressure and local interaction rules jointly contribute to emergent collective capabilities in artificial agent societies.
