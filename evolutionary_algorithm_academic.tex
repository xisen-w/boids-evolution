\subsection{Evolutionary Algorithm Module with Dynamic Specialization Discovery}

The evolutionary algorithm module introduces a novel approach to agent specialization evolution through \textit{prompt-level genetic operations} enhanced with \textit{behavioral discovery}. Unlike traditional evolutionary algorithms that operate on bit strings or numerical parameters, our implementation performs crossover and mutation directly in the semantic space of natural language specializations, leveraging Large Language Models (LLMs) as both behavioral analysts and genetic operators.

\subsubsection{Semantic Evolution Framework with Dynamic Specialization Discovery}

The core innovation lies in treating agent specialization prompts as evolvable genotypes while enabling \textit{emergent specialization discovery} from actual behavioral patterns. Each agent possesses a \texttt{specific\_prompt} field that defines its behavioral specialization, but this specialization is dynamically refined based on the agent's demonstrated capabilities and tool-building patterns before each evolutionary cycle.

\textbf{Genotype-Phenotype Distinction:} The system maintains a clear separation between genetic and phenotypic information:
\begin{itemize}
    \item \textbf{Genotype (Evolved):} \texttt{specific\_prompt} - the agent's specialization directive that undergoes crossover, mutation, and behavioral refinement
    \item \textbf{Phenotype (Recorded):} \texttt{reflection\_history} - behavioral memory that informs specialization discovery but does not directly evolve
\end{itemize}

\textbf{Dynamic Specialization Discovery:} Before each evolutionary cycle, agents undergo comprehensive behavioral analysis where their \texttt{specific\_prompt} is updated based on demonstrated competencies. The discovery process employs a sophisticated LLM-mediated analysis framework examining four dimensions:
\begin{enumerate}
    \item \textbf{Core Competencies:} Types of tools successfully created, problem-solving approaches favored, and technical domains gravitated toward
    \item \textbf{Unique Perspective:} Distinctive methodologies, unique angles brought to problems, and emergent patterns in reflection style
    \item \textbf{Collaborative Role:} Contribution to the meta-prompt objective, ecosystem niche filled, and complementary relationships with other agents
    \item \textbf{Evolutionary Potential:} Adjacent domains for exploration, untapped capabilities suggested by behavior, and directions for mutation/crossover
\end{enumerate}

This behavioral analysis transforms static initial specializations into dynamic, evidence-based specializations that accurately reflect each agent's emergent capabilities and role within the collaborative ecosystem. Evolution then operates on these discovered specializations rather than arbitrary initial assignments.

\textbf{Fitness Evaluation:} Agent fitness is determined by the average Tool Complexity Index (TCI) of all tools the agent has successfully created. The TCI metric combines code complexity, interface sophistication, and compositional reuse, providing an objective measure of an agent's contribution to ecosystem capability. Agents with no successfully created tools receive a fitness score of 0.0.

\textbf{Selection Mechanism:} The algorithm employs truncation selection, eliminating the bottom 20\% of agents based on fitness ranking while maintaining a minimum population size threshold. This approach provides consistent selection pressure while preventing population collapse in small experimental groups.

\subsubsection{Prompt-Level Genetic Operations}

\textbf{Crossover (Sexual Reproduction):} When two parent agents are selected for reproduction, their specialization prompts are combined through an LLM-mediated process. The system prompt instructs the LLM to "create a new agent specialization that combines the best aspects of both parents," while the user prompt provides the parent specializations and their performance metrics. This generates semantically coherent hybrid specializations that blend parental traits. Temperature is set to 0.7 to balance creativity with consistency.

\textbf{Mutation (Asexual Reproduction):} Single-parent reproduction creates variations through semantic mutation. The LLM is prompted to "create a related but distinct specialization that explores a different angle or approach" based on the parent's discovered specialization. Higher temperature (0.8) encourages greater exploration while maintaining thematic coherence with the parent specialization.

\subsubsection{Behavioral Analysis Methodology}

The dynamic specialization discovery process represents a novel integration of behavioral analysis with evolutionary computation. Each agent's behavioral profile is constructed from:

\textbf{Tool Creation Patterns:} Analysis of tool names, descriptions, and TCI complexity scores to identify technical preferences and capability trajectories. Tools are examined for recurring themes, complexity evolution, and domain specialization trends.

\textbf{Reflection Analysis:} Natural language processing of agent reflection histories to identify problem-solving approaches, strategic thinking patterns, and collaborative insights. This reveals cognitive styles and strategic preferences that may not be apparent from tool creation alone.

\textbf{Contextual Integration:} The meta-prompt serves as collaborative context, ensuring discovered specializations align with broader ecosystem objectives while maintaining individual agent distinctiveness.

\textbf{Prompt Engineering Framework:} The discovery process employs a comprehensive prompt structure that:
\begin{itemize}
    \item Establishes collaborative context through the shared meta-prompt
    \item Provides behavioral evidence (tools, reflections, performance metrics)
    \item Guides analysis through structured dimensions (competencies, perspective, role, potential)
    \item Generates evolution-ready specializations suitable for crossover and mutation
\end{itemize}

This methodology enables true emergent specialization where agents develop authentic expertise areas based on demonstrated performance rather than pre-assigned roles, creating more realistic and effective evolutionary dynamics.

\subsubsection{Integration with Boids Coordination}

The evolutionary algorithm operates as a complementary mechanism to boids rules rather than a replacement. During normal operation, agents follow local neighborhood coordination rules (separation, alignment, cohesion) for immediate tool creation decisions. Evolution intervenes periodically (every $n$ rounds, configurable) to update agent specializations based on accumulated performance data.

This dual-mechanism approach enables both short-term local coordination and long-term global optimization. Boids rules ensure immediate ecosystem coherence and diversity, while evolutionary pressure drives specialization refinement and capability enhancement over multiple generations.

\subsubsection{Evolutionary Dynamics}

\textbf{Timing:} Evolution triggers at configurable intervals (default: every 5 rounds) during the final rounds of experiments to allow sufficient performance data accumulation. The condition \texttt{round\_num \% evolution\_frequency == 0} determines trigger timing.

\textbf{Population Management:} New agents are created with evolved specializations but retain all other characteristics (Azure client, tool registry access, environment capabilities). Agent IDs follow the pattern \texttt{EvoAgent\_Gen\{N\}\_\{i\}\_\{Type\}} where N is the generation number, i is the agent index, and Type indicates "Cross" (crossover) or "Mut" (mutation).

\textbf{Fallback Mechanisms:} Robust error handling ensures system stability when LLM calls fail. Failed crossover attempts default to descriptive hybrid specializations, while failed mutations default to variant descriptions. This prevents evolution failure from disrupting the broader experimental framework.

\subsubsection{Data Collection and Analysis}

The system maintains comprehensive evolutionary records including generation numbers, population sizes, fitness distributions, eliminated agent IDs, and complete specialization genealogies. Evolution history enables post-hoc analysis of specialization trajectories, fitness landscapes, and convergence patterns.

Performance metrics track complexity improvement over generations, total agents eliminated and created, and diversity measures across the evolving population. This data supports both immediate experimental analysis and broader theoretical insights into semantic evolution dynamics.

\subsubsection{Experimental Validation and Empirical Results}

The modular design enables systematic comparison across multiple experimental conditions: (1) boids coordination only, (2) evolutionary selection only, (3) combined boids and evolution, (4) discovery-enhanced evolution, and (5) control condition with neither mechanism. This factorial design isolates the contributions of local coordination, global selection pressure, and behavioral discovery to collective intelligence emergence.

\textbf{Discovery Evolution Results:} Initial experimental validation using 4 agents over 7 rounds with evolution frequency of 3 rounds demonstrated successful specialization discovery and refinement. The system successfully:
\begin{itemize}
    \item Generated 28 total tools across 2 evolutionary generations
    \item Achieved 85.7\% test pass rate with 100\% test coverage
    \item Eliminated 2 agents and created 2 evolved agents with refined specializations
    \item Demonstrated emergent specialization in domains including trust orchestration, data transformation, and mathematical optimization
\end{itemize}

\textbf{Behavioral Discovery Validation:} The system successfully identified emergent specializations such as "Trust Orchestration Layer" and "Context-Aware Module Suggestion Engine," demonstrating that agents developed authentic expertise areas beyond their initial generic assignments. These discovered specializations showed clear alignment with actual tool creation patterns and collaborative ecosystem needs.

\textbf{Evolutionary Dynamics:} The dual-phase evolution process (discovery followed by genetic operations) created more sophisticated and contextually appropriate agent specializations compared to traditional crossover/mutation alone. Agents evolved from generic initial prompts to specific, behaviorally-grounded specializations that reflected demonstrated capabilities.

By introducing evolutionary pressure into the semantic space of agent specializations while enabling dynamic behavioral discovery and preserving local coordination mechanisms, this approach represents a novel synthesis of swarm intelligence, evolutionary computation, and behavioral analysis principles. This enables investigation of how global selection pressure, local interaction rules, and emergent specialization discovery jointly contribute to authentic collective capabilities in artificial agent societies.
