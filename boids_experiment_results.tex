\documentclass{article}
\usepackage{booktabs}
\usepackage{array}
\usepackage{multirow}

\begin{document}

\section{Boids Evolution Experiment Results}

\begin{table}[htbp]
\centering
\caption{Comparative Analysis: Boids vs Baseline Agent Societies in Data Science Tool Ecosystem Development}
\label{tab:boids_results}
\begin{tabular}{@{}lcc@{}}
\toprule
\textbf{Metric} & \textbf{Boids Experiment} & \textbf{Baseline Experiment} \\
\midrule
\textbf{Experimental Configuration} & & \\
Agents & 10 & 20 \\
Rounds & 10 & 15 \\
Network Topology & k-regular ring (k=2) & Fully connected \\
Decision Making & Local neighborhood rules & Global reflection \\
\midrule
\textbf{Tool Ecosystem Metrics} & & \\
Total Tools Created & 17 & 21 \\
Tools per Agent & 1.7 & 1.05 \\
Average Tool Complexity (LoC) & 50-75 & 100+ \\
Tool Specialization & High (focused modules) & Low (monolithic pipelines) \\
\midrule
\textbf{Emergent Behavior Evidence} & & \\
Infrastructure Tools & 3 (rate limiting) & 1 (rate limiting) \\
Functional Redundancy & Low & High \\
Complementary Tools & 14/17 (82\%) & 8/21 (38\%) \\
\midrule
\textbf{Collaboration Indicators} & & \\
Tool Name Alignment & High (json\_*, data\_*) & Medium \\
Interface Consistency & High & Low \\
Separation Compliance & 100\% (no duplicates) & 76\% \\
\midrule
\textbf{Quality Metrics} & & \\
Average TCI Score & 2.1-3.8 & 3.0-11.3 \\
Test Pass Rate & 94\% & 89\% \\
Functional Coverage & Comprehensive & Overlapping \\
\bottomrule
\end{tabular}
\end{table}

\subsection{Analysis and Insights}

The experimental results provide compelling evidence for the efficacy of Boids-inspired swarm dynamics in fostering emergent intelligence within AI agent societies. Most remarkably, the spontaneous convergence on rate limiting infrastructure across three independent agents (RateLimitManager and RateLimitMonitor variants) represents genuine emergent problem-solving behavior. This phenomenon demonstrates how local neighborhood awareness and simple interaction rules can lead agents to collectively identify and address systemic challenges without explicit coordination or centralized planning. The Boids experiment achieved superior tool ecosystem diversity and specialization despite utilizing only half the agents and two-thirds the rounds of the baseline, suggesting that constrained local interactions paradoxically enhance rather than limit collective intelligence. The 82\% complementary tool rate in the Boids condition versus 38\% in the baseline further supports the hypothesis that separation, alignment, and cohesion rules effectively guide agents toward non-redundant, synergistic contributions to the shared ecosystem.

Furthermore, the qualitative differences in tool architecture reveal deeper insights into how decision-making frameworks shape emergent system properties. Boids agents consistently produced modular, focused tools averaging 50-75 lines of code with clear single responsibilities, while baseline agents gravitated toward monolithic ``end-to-end'' solutions exceeding 100 lines with complex, overlapping functionality. This architectural divergence suggests that local neighborhood constraints encourage agents to develop specialized competencies that complement rather than duplicate their peers' contributions. The higher tools-per-agent productivity (1.7 vs 1.05) combined with lower average complexity scores (2.1-3.8 vs 3.0-11.3 TCI) indicates that Boids dynamics promote efficient resource utilization and sustainable ecosystem growth. These findings validate the core premise that simple, biologically-inspired interaction rules can generate sophisticated collaborative behaviors in artificial agent societies, offering promising directions for designing scalable multi-agent systems that exhibit emergent intelligence through decentralized coordination mechanisms.

\end{document} 